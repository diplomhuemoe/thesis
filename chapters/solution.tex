% -*-coding: utf-8-*-
\chapter{Исследование и построение решения задачи}

Основной задачей, поставленной в данной работе является разработка метода
построения словаря эмоционально окрашенных слов на основе алгоритма
распространяющейся активации. Соответственно решение данной задачи включает в
себя следующие пункты:

\begin{enumerate}
  \item Теоретическое исследование применимости метода распространяющейся
    активации для задачи построения словаря эмоционально окрашенных слов;
  \item Реализация метода распространяющейся активации и его вариантов;
  \item Экспериментальное сравнение реализованных алгоритмов.
\end{enumerate} 

\section{Теоретическое исследование применимости метода распространяющейся
  активации}
\subsection{Алгоритм распространяющейся активации}
Пусть задан граф с $N$ вершинами, каждой из которых сопоставлено значение
активации $A[i]$, которое является вещественным числом на отрезке $[0; 1]$.
Каждое ребро $[i, j]$, соединяющее вершины $i$ и $j$ имеет вес, обозначаемый как
$W[i, j]$, который тоже является вещественным числом с отрезка $[0; 1]$. Также
задан коэффициент затухания D, имеющий вещественное значение с отрезка $[0; 1]$.

Процедура:
\begin{enumerate}
\item Все значения активации $[i]$ обнуляются. Выбирается несколько вершин, с
  которых начнется активация и устанавливается их $A[i]$;
\item Для каждой вершины, имеющей ненулевое $A[i]$, рассчитывается значение
  активации всех его соседних вершин(которые соединены с ним ребром $[i, j]$) по
  следующей формуле $A[j] = max(A[j]; A[i] * W[i, j] * D)$;
\item Пункт 2 повторяется до тех пор пока суммарное значение изменения активации
  по всем вершинам не станет меньше какого то установленного малого значения.
\end{enumerate}

\subsection{Применение метода распространяющейся активации для задачи построения
  словаря эмоционально окрашенных слов}
Для построения необходимого графа слова выступают как вершины, а наличие
ребра между словами означает существования биграммы из них в текстовом корпусе.
Каждая вершина при этом уникальна и в итоге получается граф всех словосочетаний,
считанных из набора данных. Веса ребер для каждой биграммы также заданы
как какая-либо её характеристика. Например, это может быть количество вхождения
биграммы в текст.

Для каждой вершины в графе будет два значения активации: первое (будем его
называть $P[i]$) является вероятностью принадлежности слова к классу позитивно
окрашенных, второе значение (назовем его $N[i]$) - это вероятность принадлежности
к классу негативно окрашенных слов. Для инициализации вручную составляется
два списка слов. В первом списке слова придающие тексту заведомо позитивную
окраску, поэтому их $P[i]$ устанавливается равным единице при инициализации графа.
Во втором списке слова придающие негативную окраску, их $N[i]$ становятся равны
также единице. Оба списка представлены в Приложении $1$. После этого на графе
запускается распространяющаяся активация.

\FloatBarrier

%%% Local Variables:
%%% mode: latex
%%% TeX-master: "../main"
%%% End:
