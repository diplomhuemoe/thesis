% -*-coding: utf-8-*-
\startprefacepage

Текстовую информацию можно разделить на две различных категории: факты
и мнения. Факты являются объективными высказываниями о некоторых событиях
или сущностях. Мнения являются субъективными высказываниями, которые обыч-
но выражают отношение людей к различным событиям, явлениям или сущностям.
Но если мнений на какую то тему довольно большое количество, то, проанализиро-
вав их соответствующим образом, можно выяснить объективную оценку изучаемого
события или сущности. Таким образом, на основе множества мнений можно синте-
зировать новые факты, что, несомненно, полезно и достойно исследования.

С развитием и распространением Интернета появляется все больше и больше
мнений, выражаемых самыми разными людьми. Это, к примеру, отзывы о товарах в
интернет-магазинах, реакция на мировые события в блогах, социальных сетях и на
интернет-форумах. В результате получается очень объемный контент, создаваемый
всеми пользователями. Сейчас, когда человек хочет приобрести какой то продукт,
ему уже не так необходимо узнавать мнение своих друзей на этот счет. Достаточно
лишь почитать отзывы в интернете и сделать вывод о качестве предстоящей покупки.

С другой стороны производителям и продавцам тоже необходимо понимать отношение
пользователей к их продуктам или услугам. Для таких целей всегда традиционно
нанимались консультанты или проводились соответствующие опросы. Однако, мнений
сейчас становится очень много, так как число их источников постоянно
растет, поэтому все больше времени и средств тратится на их анализ.

Отсюда и появляется задача автоматического анализа тональности текста - это
задача определения эмоционального отношения автора текста к некоторому объекту
(объекту реального мира, событию, процессу или их свойствам/атрибутам),
выраженному в тексте. Тональность всего текста в целом можно определить как
функцию (например, сумму) лексических тональностей составляющих его единиц
(предложений, слов) и правил их сочетания. Соответственно, в простейшем случае
тональная оценка может быть позитивная или негативная.

Методов решения этой задачи достаточно много, потому как мнения бывают
разных видов: например, к анализу статей в журналах и к сообщениям в социальных
сетях с неформальной лексикой нужны совсем разные подходы. Последние в свою
очередь очень сильно набирает популярность, ежедневно размещаются миллионы
сообщений обычных пользователей с суждениями о том с чем они сталкиваются в
повседневной жизни.

Примером такой социальной сети является Twitter — социальная сеть для публичного
обмена короткими (до 140 символов) сообщениями, которой пользуются сотни
миллионов пользователей. Что очень важно для данного исследования: случайные
twitter-сообщения редко имеют логическую связь между собой и пишутся на самые
разнообразные темы. Это делает его очень удобным для тестирования новых
реализаций методов анализа тональности.

В настоящей дипломной работе рассматривается задача построения словаря
эмоционально окрашенных слов и последующее его применения для анализа
тональности текстового корпуса.

В главе 1 будет рассмотрена постановка задачи и предоставлен обзор существующих
решений.

В главе 2 будет проведено исследование и построение решения задачи.

В главе 3 будет описана практическая часть: инструменты разработки, общая схема
работы и архитектура системы.

В главе 4 будут предоставлены сравнения с существующими решениями на конкретных
примерах и сделаны выводы по эффективности предложенного алгоритма.

\FloatBarrier

%%% Local Variables:
%%% mode: latex
%%% TeX-master: "../main"
%%% End:
