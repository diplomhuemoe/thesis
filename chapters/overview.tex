% -*-coding: utf-8-*-
\chapter{Постановка задачи и обзор существующих решений}
\section{Постановка задачи}
Целью данной дипломной работы является исследование и разработка метода
распространяющейся активации для составления словаря эмоционально окрашенных
слов на русском языке, а также последующая оценка эффективности его применения.
Тестирование метода будет производиться уже на размеченном текстовом корпусе
с равным количеством позитивных и негативных примеров. Мерой точности будет
выступать доля верно распознанной эмоциональной окраски. Для достижения цели
необходимо решить следующие задачи:
\begin{enumerate}
  \item Исследовать предметную область, изучить существующие методы
    автоматического анализа тональности текстов;
  \item Провести теоретическое исследование применимости метода
    распространяющейся активации для задачи построения словаря эмоционально
    окрашенных слов;
  \item Реализовать метод распространяющейся активации и его варианты с целью
    решения задачи построения словаря эмоционально окрашенных слов;
  \item Произвести экспериментальное сравнения реализованных алгоритмов.
\end{enumerate}

\section{Обзор существующих решений}
Сделав большое обобщение, можно разделить существующие подходы на следу-
ющие категории:

\begin{enumerate}
  \item Подходы, основанные на словарях
  \item Подходы, основанные на машинном обучении
  \item Подходы, основанные на правилах
\end{enumerate} 

\subsection{Использование словарей}
Такой подход использует так называемый тональный словарь (affective lexicon)
для анализа текста. В простом виде в тональном словаре словам ставится в
соответствие вероятность быть отнесенным к определенной эмоции. Эта
вероятность обычно рассчитывается на основе какого-либо текстового корпуса.

Этот подход имеет свои недостатки:

\todo{Дописать}

\FloatBarrier

%%% Local Variables:
%%% mode: latex
%%% TeX-master: "../main"
%%% End:
