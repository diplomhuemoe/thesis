% -*-coding: utf-8-*-
\chapter{Постановка задачи и обзор существующих решений}
\section{Постановка задачи}
Целью данной дипломной работы является исследование и разработка метода
распространяющейся активации для составления словаря эмоционально окрашенных
слов на русском языке, а также последующая оценка эффективности его применения.
Тестирование метода будет производиться уже на размеченном текстовом корпусе
с равным количеством позитивных и негативных примеров. Мерой точности будет
выступать доля верно распознанной эмоциональной окраски. Для достижения цели
необходимо решить следующие задачи:
\begin{enumerate}
  \item Исследовать предметную область, изучить существующие методы
    автоматического анализа тональности текстов;
  \item Провести теоретическое исследование применимости метода
    распространяющейся активации для задачи построения словаря эмоционально
    окрашенных слов;
  \item Реализовать метод распространяющейся активации и его варианты с целью
    решения задачи построения словаря эмоционально окрашенных слов;
  \item Произвести экспериментальное сравнения реализованных алгоритмов.
\end{enumerate}

\section{Обзор существующих решений}
Сделав большое обобщение, можно разделить существующие подходы на следу-
ющие категории:

\begin{enumerate}
  \item Подходы, основанные на словарях
  \item Подходы, основанные на машинном обучении
  \item Подходы, основанные на правилах
\end{enumerate} 

\subsection{Использование словарей}
Такой подход использует так называемый тональный словарь (affective lexicon)
для анализа текста. В простом виде в тональном словаре словам ставится в
соответствие вероятность быть отнесенным к определенной эмоции. Эта
вероятность обычно рассчитывается на основе какого-либо текстового корпуса.

Этот подход имеет свои недостатки:

\begin{itemize}
\item Одно слово может иметь несколько значений, имеющих сильно различающиеся
тональности; 
\item Результаты будут плохими, если анализировать текст такого жанра, который
сильно отличается по языковым свойствам от жанра текстового корпуса, на
основе которого составлялся словарь;
\end{itemize}

Тем не менее этот подход очень интересен для исследования: можно использо-
вать различные текстовые корпуса и применять различные алгоритмы к ним для
построения словарей. Именно этот подход изучается в данной дипломной работе.

Чтобы проанализировать текст, основываясь на этом подходе, можно воспользоваться
следующим алгоритмом: сначала каждому слову в тексте присвоить его
значением тональности из словаря (если оно присутствует в словаре), а затем
вычислить общую тональность всего текста. Вычислять общую тональность можно
разными способами. Самый простой из них -- среднее арифметическое всех значений.
Более сложный -- обучить классификатор.

Стоит рассмотреть несколько известных словарей, специально размеченных с
учётом эмоциональной составляющей.

\subsection{WordNet-Affect}
WordNet -- это электронный тезаурус для английского языка, разработанный в
Принстонском университете. Базовой словарной единицей в WordNet является не
отдельное слово, а так называемый синонимический ряд (синсеты), объединяющий
слова со схожим значением и по сути своей являющимися узлами сети.

WordNet-Affect был создан на основе WordNet для английского языка путём выбора и
отнесения синсетов к различным эмоциональным понятиям. Синсеты основных
частей речи были вручную размечены специальными эмоциональными метками, которые
характеризуют различные состояния, выражающие эмоциональные
отклики, или ситуации, которые вызывают эмоции. Все такие метки обьединяются
в четыре дополнительных эмоциональных метки: позитивная, негативная,
неоднозначная и нейтральная.

Физическая структура WordNet-Affect состоит из шести файлов-категорий: радость,
страх, гнев, печаль, отвращение, удивление. На данный момент в этом словаре
около 2900 синсетов и 4800 слов.

\subsection{SentiWordNet}
SentiWordNet - это словарь, полученный посредством автоматического аннотирования
синсетов из WordNet в соответствии с его степенью позитивности, негативности и
объективности. Каждая из этих степеней оценивается значением из интервала (0;
1), причем все три в сумме должны давать 1.

Процесс создания SentiWordNet состоял из двух шагов:
\begin{enumerate}
\item Используются методы машинного обучения с частичным привлечением учителя.
  Вначале выбиралось небольшое число синсетов, которые размечались вручную.
  Затем на них было обучено несколько классификаторов, которые должны
  были определять численные оценки каждого из синсетов. Таким образом через
  полученные модели были размечены все оставшиеся синсеты. 
\item Затем к данным применялась модель случайного блуждания, чтобы установить
  окончательные оценки объективной, позитивной или негативной составляющей
  каждого синсета
\end{enumerate}

\subsection{SenticNet}
SenticNet это еще один семантический тезаурус. Его отличие от двух рассмотренных
состоит в том, что WordNet-Affect и SentiWordNet обеспечивают связывание
слов и эмоциональных понятий на синтаксическом уровне, а SenticNet связывает
понятия на семантическом уровне.

SenticNet построен на основе так называемых "sentic-вычислений". Это парадигма,
которая использует методы искусственного интеллекта и семантической паутины для
обработки мнений на естественном языке. Такая концепция позволяет
проводить анализ документов не только на уровне целых страниц и текстов, но и
на уровне предложений, что позволяет оценивать тексты на более высоком уровне
детализации.

SenticNet сопоставляет каждому понятию Sentic-вектор"с численными значениями
таких величин, как приятность, внимание, чуткость и способность, а также
величину тональности, для задачи анализа тональности текста. На данный момент в
этом тезаурусе около 14000 понятий.

\FloatBarrier

%%% Local Variables:
%%% mode: latex
%%% TeX-master: "../main"
%%% End:
